\usepackage{listings} % coding environment (used to show verilog codes)

% Put custom commands here

\makeatletter
% ==========================================
% Define custom commands:

% This is a Zebra code listing (creates neat ``zebra''-like
% lines with numberings. You can define your own colors
% (modify the code_params.tex)
\newcommand\realnumberstyle[1]{#1}
\newcommand{\zebra}[3]{%
  {\realnumberstyle{#3}}%
  \begingroup
    \lst@basicstyle
    \ifodd\value{lstnumber}%
        \color{#1}%
    \else
        \color{#2}%
    \fi
    \rlap{%
        \hspace*{\lst@numbersep}%
        \color@block{\linewidth}{\ht\strutbox}{\dp\strutbox}%
    }%
  \endgroup
}

% This one is for making spaces around the wrapfigures
% sometimes when the text is short, the wrapfigure will
% try to make everything after itself change its size.
% use \makespace to fix it
\newcommand{\makespace} {
  \par
  \loop
  \ifnum\c@WF@wrappedlines > 1
  \hspace*{1sp}\newline
  \advance\c@WF@wrappedlines by -1
  \repeat
  \par
}


%-------------------------------------------------------------------------
% Extra definitions
%-------------------------------------------------------------------------
% If you would like to get rid of some text, but still have it for future:
% Just put the text that you would like to hide into \ignore{your text}
\def\ignore#1{}
% \newcommand{\ignore}[1]{}

% This is a fixme request. In case of collaboration, this thing will emphasize
% the text and put a FIXME: keyword
% \newcommand{\fixme}[1] {
%  \textbf{\textcolor{red}{[ FIXME: #1 ]}}
% }
\newcommand{\fixme}[1] {
  \todo[backgroundcolor=red!50!white,inline,author=FIXME]{#1}
}

% These are the todo requests
\newcommand{\noteforme}[1] {
  \textbf{\textcolor{blue}{[ NOTE FOR MYSELF: #1 ]}}
}
\newcommand{\todoBlue}[1] {
  \todo[backgroundcolor=blue!20!white,inline,author=TODO]{#1}
}
\newcommand{\todoRed}[1] {
  \todo[backgroundcolor=red!20!white,inline,author=TODO]{#1}
}
\newcommand{\todoGreen}[1] {
  \todo[backgroundcolor=green!20!white,inline,author=TODO]{#1}
}
\newcommand{\todoYellow}[1] {
  \todo[backgroundcolor=yellow!20!white,inline,author=TODO]{#1}
}

% Create tabs
\newcommand{\tab} {
  \hspace{.02\textwidth}
}

% End of commands definition
% ==========================================
\makeatother

% ==========================================
% Define code styles:

% Before defining the verilog code listing, let's fix the referencing
% Use matchrangestart=t as an argument to the lslisting to enable
\makeatletter
\lst@Key{matchrangestart}{f}{\lstKV@SetIf{#1}\lst@ifmatchrangestart} % Define matchrangestart
% Add matchrangestart test to the SkipToFirst:
\def\lst@SkipToFirst{%
    \lst@ifmatchrangestart\c@lstnumber=\numexpr-1+\lst@firstline\fi
    \ifnum \lst@lineno<\lst@firstline
           \def\lst@next{%
                \lst@BeginDropInput\lst@Pmode
                \lst@Let{13}\lst@MSkipToFirst
                \lst@Let{10}\lst@MSkipToFirst}%
           \expandafter\lst@next
    \else
        \expandafter\lst@BOLGobble
    \fi}%
\makeatother

% Verilog styling with Zebra
\lstdefinestyle{custom_verilog}{
  basicstyle=\footnotesize\ttfamily,
  belowcaptionskip=0.1\baselineskip,
  breaklines=true,
  captionpos=t,
  commentstyle=\color{green!60!black},
  deletekeywords={...},
  % Note that @ sign is better to use as an escape character
  % but in Verilog it is already reserved:
  escapeinside={^}{^},
  frame=L,
  identifierstyle=\color{black},
  keywordstyle=\bfseries\color{blue},
  language=Verilog,
  morecomment={
    [s][\color{red}]{/*!}{*}
  },
  morekeywords={
    *,
    % Data type definitions:
    bit,
    electrical,
    logic,
    longint,
    shortint,
    shortreal,
    typedef,
    wreal,
    ...,
    % extra constructs:
    interface,
    endinterface,
    exclude,
    from,
    ...,
    % blocks:
    analog,
    ...},
  numbers=left,
  numberstyle=\zebra{gray!20}{white}, % Don't forget to declare ZEBRA command
  showstringspaces=false,
  stepnumber=1,
  stringstyle=\color{orange},
  tabsize=2,
  % title=\lstname,
  xleftmargin=\parindent,
}

%%
% Change some of the code params:
\renewcommand{\lstlistingname}{Code Listing} % Change the listing caption
\lstset{style=custom_verilog} % By default assume all codes are verilog type


% End of code styles
% ==========================================
